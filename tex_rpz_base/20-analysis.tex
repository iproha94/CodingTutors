\chapter{Аналитический раздел}
Первый шаг работы поисковой системы~--- сбора документов. Для этого применяются поисковые роботы, задачей которых является нахождения страниц в сети. Робот сканирует страницы на предмет ссылок на другие страницы, после чего процесс повторяется заново.

Далее собранные документы необходимо проиндексировать, то есть перевести их в форму, удобную для дальнейшего поиска. В зависимости от конкретного приложения сами документы могут и не храниться в системе, тогда в индексе находится лишь ссылка.

Ну и последний шаг~--- это, конечно, возврат ранжированного списка документов в ответ на запрос пользователя.

Поскольку именно поиск документов является главной целью, система должна разрабатываться с учётом тех характеристик документов, которые участвуют в ранжировании, так как от этого зависит способ сбора и хранения данных.


% \input{21-ranking}
% \input{22-collecting}
